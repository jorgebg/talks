% last updated in April 2002 by Antje Endemann
% Based on CVPR 07 and LNCS, with modifications by DAF, AZ and elle, 2008 and AA, 2010, and CC, 2011; TT, 2014; AAS, 2016

\documentclass[runningheads]{llncs}
\usepackage[english,spanish.notilde.lcroman,shorthands=off]{babel}
\usepackage{graphicx}
\usepackage{amsmath,amssymb} % define this before the line numbering.
\usepackage{ruler}
\usepackage{color}
\usepackage{enumitem,amssymb}
\usepackage{url}
\newlist{todolist}{itemize}{2}
\setlist[todolist]{label=$\square$}
\usepackage{pifont}
\newcommand{\cmark}{\ding{51}}%
\newcommand{\done}{\rlap{$\square$}{\raisebox{2pt}{\large\hspace{1pt}\cmark}}%
\hspace{-2.5pt}}
\usepackage[width=122mm,left=12mm,paperwidth=146mm,height=193mm,top=12mm,paperheight=217mm]{geometry}

\usepackage{fancyhdr}
\fancyhf{}
\renewcommand{\headrulewidth}{0pt}
\renewcommand{\footrulewidth}{0pt}
\setlength\headheight{80.0pt}
\addtolength{\textheight}{-80.0pt}
%\chead{\includegraphics[width=0.15\textwidth]{Largo.PNG}}

\begin{document}

\pagestyle{headings}
\mainmatter


\title{Service Oriented Architecture with PySOA}


\author{Jorge Barata}

\maketitle

\section{Tipo de Contribuci\'on}

\begin{todolist}
  \item [\done] Charla corta
  \item Charla extendida
  \item P\'oster
  \item Taller
  \end{todolist}


\section{Idioma}
\begin{todolist}
  \item Espa\~nol
  \item [\done] Ingl\'es
\end{todolist}
\section{Nivel}

\begin{todolist}
  \item Avanzado
  \item [\done] Medio
  \item Iniciaci\'on
  \end{todolist}

\keywords{SOA, PySOA, Redis, Microservices}

\newpage

\section{Resumen}
\textit{
PySOA is a Python library for writing (micro)services and their clients. It is based on Redis BLPOP and RPUSH commands. In this talk we’ll learn how it works under the hood.
}

\section{Presentaci\'on}

\begin{itemize}
  \item Introduction
  \item What is SOA? \cite{IanGorton}
  \item Monolith vs SOA \cite{MartinFowler}
  \begin{itemize}
    \item Benefits and Drawbacks
  \end{itemize}
  \item PySOA \cite{PySOA}
  \begin{itemize}
    \item Redis
    \item BLPOP and RPUSH
    \item MessagePack
    \item Client/Server Example \cite{PySOAExample}
    \item Testing
  \end{itemize}
  \item Conclusion
\end{itemize}



\section{Pre-requisitos para atender a la presentación}
Los asistentes deber\'an tener conocimientos previos de Python.

\section{Otros requerimientos t\'ecnicos}

Utilizar\'e mi ordenador para la propuesta y necesitar\'e adaptador HDMI a USB-C.


\clearpage

\bibliographystyle{splncs}
\bibliography{egbib}
\end{document}
